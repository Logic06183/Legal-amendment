\documentclass[12pt,letterpaper]{article}
\usepackage{geometry}
\geometry{margin=1in}
\usepackage{titlesec}
\usepackage{enumitem}
\usepackage{hyperref}
\hypersetup{
    colorlinks=true,
    linkcolor=blue,
    filecolor=magenta,
    urlcolor=blue,
}

\titleformat{\section}
  {\normalfont\Large\bfseries}{\thesection}{1em}{}
\titleformat{\subsection}
  {\normalfont\large\bfseries}{\thesubsection}{1em}{}

\setlist{noitemsep}

\title{\textbf{ANNEXURES TO WITS HEALTH CONSORTIUM\\DATA TRANSFER AGREEMENT FOR RP2}}
\author{}
\date{}

\begin{document}

\maketitle

\section*{ANNEXURE A:}

\textbf{DESCRIPTION OF DATA}

The list of variables below is indicative. Based on data availability and relevance, the final variable list shall be finalized and recorded between the Data Provider and Data Recipient.

\textbf{Data Source 1}

\begin{itemize}
    \item \textbf{Project Title}: [Full research project title]
    \item \textbf{Funder}: [Original research funding details]
\end{itemize}

\textbf{Data to be transferred}: Individual participant data for a limited set of variables from the original dataset/s relating to urban heat exposure and its health impacts on vulnerable populations in African cities.

\textbf{Dataset includes these important variables}:

\textbf{Essential variables:}
\begin{itemize}
    \item Unique ID (study ID and participant ID)
    \item Date of health event (e.g., hospitalization, mortality)
    \item Location, at a minimum: city of event or follow-up (data on the location of household, health facility, or study clinic are preferable)
\end{itemize}

\textbf{Health outcomes (indicative list):}
\begin{itemize}
    \item Heat-related illness or hospitalization
    \item Cardiovascular events (e.g., heart attack, stroke)
    \item Respiratory conditions exacerbated by heat
    \item Mortality (including cause of death)
    \item Admissions to intensive care units
    \item Chronic disease exacerbations (e.g., diabetes, hypertension)
\end{itemize}

\textbf{Socio-economic and environmental variables (indicative list):}
\begin{itemize}
    \item Housing type and quality (e.g., informal housing, presence of air conditioning)
    \item Access to health services
    \item Employment status (outdoor vs. indoor workers)
    \item Socio-economic status indices
    \item Neighborhood-level data (e.g., urban heat island effect, green space availability)
\end{itemize}

\textbf{Other variables:}
\begin{itemize}
    \item Age, gender, and other demographic information
    \item Date of interviews or examinations
    \item Specific location of health facility or research site
    \item Type of facility (e.g., clinic, hospital)
    \item History of chronic diseases
    \item Anthropometric data (e.g., weight, height, BMI)
\end{itemize}

\textbf{Associated metadata/documentation:}
\begin{itemize}
    \item Study protocol
    \item Codebooks
    \item Do files
    \item Documentation on definitions, components, and processing of the data
\end{itemize}

\textbf{Purpose of Data Transfer}: The data will be used to advance the understanding of heat-health interactions in large African cities and to develop an Early Warning System for urban heat-health risks.

\textbf{Data Source 2}:

[Repeat as above for each dataset to be shared]

\newpage

\section*{ANNEXURE B:}

\textbf{DESCRIPTION OF STUDY}

\textbf{Study title:} Innovative Machine Learning and multi-source data analysis towards the development of an urban heat-health Early Warning System for African cities\footnote{Jack C, Parker C, Kouakou YE, Joubert B, McAllister KA, Ilias M, Maimela G, Chersich M, Makhanya S, Luchters S, Makanga PT, Vos E, Ebi KL, Koné B, Waljee AK, Cissé G; HE\textsuperscript{2}AT Center Group. Leveraging data science and machine learning for urban climate adaptation in two major African cities: a HE\textsuperscript{2}AT Center study protocol. BMJ Open. 2024 Jun 18;14(6):e077529. doi: 10.1136/bmjopen-2023-077529. PMID: 38890141; PMCID: PMC11191804.}

\textbf{Study rationale:} African cities are experiencing rapid urbanization and significant increases in temperatures due to climate change, with dire health implications for vulnerable populations. However, data and understanding of heat-health outcomes, exposure, vulnerability, and potential solutions in these urban contexts are critically lacking. This study seeks to fill these gaps by leveraging advanced data science techniques to map and predict urban heat-health risks and inform public health interventions.

\textbf{Study objectives:} The study's overall objective is to use innovative data science approaches to quantify the impact of heat exposure on health outcomes in urban African settings and develop a geospatial Early Warning System for urban heat-health risks.

The specific objectives are:
\begin{itemize}
    \item To map intra-urban heat vulnerability and exposure across urban areas in large African cities.
    \item To develop a collaboration between the HE\textsuperscript{2}AT Center and local stakeholders, including urban planners, public health officials, and community leaders.
    \item To link health outcome data spatially and temporally with weather, environmental, and socio-economic data.
    \item To apply machine learning and other advanced analytical methods to predict adverse health outcomes related to heat exposure.
    \item To develop and implement an Early Warning System to provide targeted alerts to vulnerable populations and guide public health responses.
\end{itemize}

\textbf{Methods:} The study will systematically collect and analyze data from various sources, including longitudinal health studies, satellite imagery, and socio-economic surveys. Machine learning models will be used to identify patterns and predict heat-related health risks. The study will also involve developing a mobile application to disseminate heat warnings and collect user feedback.

\textbf{Ethical and legal considerations:} The study has been approved by the Human Research Ethics Committee of the University of the Witwatersrand, South Africa. Data privacy and security will be ensured by removing participant identifiers, data encryption, and secure data storage.

\textbf{Funding acknowledgement:} The study is funded by the Fogarty International Center and National Institute of Environmental Health Sciences (NIEHS) and OD/Office of Strategic Coordination (OSC) of the National Institutes of Health under Award Number U54 TW 012083.

\newpage

\section*{ANNEXURE C:}

\textbf{HE²AT CENTER CONSORTIUM MEMBERS AS AT THE DATE OF SIGNING THIS DATA TRANSFER AGREEMENT:}

\begin{itemize}
    \item Wits Health Consortium (Pty) Ltd, South Africa*
    \item University of Cape Town, South Africa*
    \item International Business Machines (IBM) Corporation through its Thomas J. Watson Research Center, USA*
    \item University of Peleforo Gon Coulibaly, Côte d'Ivoire*
    \item Centre for Sexual Health and HIV AIDS Research (CeSHHAR), Zimbabwe*
    \item University of Michigan, United States
    \item University of Washington, United States
\end{itemize}

*Only these HE²AT Center Consortium Members shall have access to the Consortium-Shared Data for the RP2 Study analysis purposes.

\newpage

\section*{ANNEXURE D:}

\textbf{AUTHORSHIP GUIDELINES FOR STUDIES THAT CONTRIBUTE DATA}

Study Principal Investigators, Site Principal Investigators, and additional contributing study members will be invited to join the authorship group for any publications that use data from their study.

The authorship guidelines adhere to the ICMJE criteria for authorship, which include:
\begin{itemize}
    \item Substantial contributions to the conception or design of the work or the acquisition, analysis, or interpretation of data for the work; AND
    \item Drafting the work or revising it critically for important intellectual content; AND
    \item Final approval of the version to be published; AND
    \item Agreement to be accountable for all aspects of the work in ensuring that questions related to the accuracy or integrity of any part of the work are appropriately investigated and resolved.
\end{itemize}

The authorship guidelines and study acknowledgements are based on an appreciation of the substantial contribution made by Principal Investigators in providing data from their study, and in recognition of the work involved in conducting the study.

We will include one author per included study (usually the study PI), but additional country-PI will be included for multi-country studies. The listed authors of the studies contributing data will be named in alphabetical order by surname, from positions 4th author to second-last author. As such, authorships 1-3 and last authorship will be reserved for those who contributed most to the work, in line with ICMJE guidelines.

Some journals may restrict the number of authors listed and require that additional authors be included as part of the 'HE²AT Center Study Group'. In this situation, the HE²AT Center Steering Committee will have the right to decide on final authorship, taking into consideration the studies that contributed the most participants to the RP2 dataset.

The study group will be published in an Appendix, where journals allow this or otherwise list it in the acknowledgement section. Here, listing will be done by role in the study and/or by Study/site. Any additional contributors from a study who adhere to ICMJE criteria will be listed as part of the 'HE²AT Center Study Group' in an Appendix where journals allow this or otherwise be listed in the acknowledgement section.

The name of the funder of the contributing study and of other Principal Investigators will be included in the acknowledgements as relevant.

Study Principal Investigators may be granted access to the RP2 De-Identified Data for secondary analyses, provided they complete the Data Request Forms, which will then be reviewed by the Data Access Committee (DAC). Decisions around data access are governed by the HE²AT Center's Data Management Plan and the Publication Policy Standard Operating Procedures.

\end{document}