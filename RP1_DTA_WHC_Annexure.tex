\documentclass[12pt,letterpaper]{article}
\usepackage{geometry}
\geometry{margin=1in}
\usepackage{titlesec}
\usepackage{enumitem}
\usepackage{hyperref}
\hypersetup{
    colorlinks=true,
    linkcolor=blue,
    filecolor=magenta,
    urlcolor=blue,
}

\titleformat{\section}
  {\normalfont\Large\bfseries}{\thesection}{1em}{}
\titleformat{\subsection}
  {\normalfont\large\bfseries}{\thesubsection}{1em}{}

\setlist{noitemsep}

\title{\textbf{ANNEXURES TO WITS HEALTH CONSORTIUM\\DATA TRANSFER AGREEMENT FOR RP1}}
\author{}
\date{}

\begin{document}

\maketitle

\section*{ANNEXURE A:}

\textbf{DESCRIPTION OF DATA}

The below list of variables is indicative, and the final variable list shall be finalised and recorded between Data Provider and Data Recipient based on data availability and relevance.

\textbf{Data Source 1}

\begin{itemize}
    \item \textbf{Project Title}: [Full research project title]
    \item \textbf{Funder}: [Original research funding details]
\end{itemize}

\textbf{Data to be transferred:} Individual participant data for a limited set of variables from the original dataset/s relating to Maternal outcomes and/or fetal, neonatal and child outcomes.

\textbf{Dataset includes these important variables:}

\textbf{Essential variables:}
\begin{itemize}
    \item Unique ID (study ID and participant ID)
    \item Date of delivery of the newborn OR date of maternal outcomes
    \item Location, at a minimum: city of delivery, or city of follow-up (data on location of household, birth facility, or study clinic are preferable)
\end{itemize}

\textbf{Maternal outcomes (indicative list):}
\begin{itemize}
    \item Gestational age at delivery
    \item Preterm premature rupture of the membranes
    \item Prolonged rupture of membranes
    \item Antepartum and postpartum haemorrhage
    \item Hypertensive disorders in pregnancy
    \item Anaemia in pregnancy
    \item Adverse events
    \item Gestational Diabetes Mellitus
    \item Health facility visits
    \item Maternal mental health
\end{itemize}

\textbf{Fetal, neonatal and child outcomes (indicative list)}
\begin{itemize}
    \item Prematurity (see also gestational age at delivery)
    \item Mortality (including cause)
    \item Mother-to-child transmission of HIV (MTCT)
    \item APGAR score
    \item Infant growth
    \item Admission to neonatal intensive care units or paediatric ward
    \item Intrauterine growth restriction
\end{itemize}

\textbf{Other variables}
\begin{itemize}
    \item Maternal age
    \item Date of interviews or examination
    \item Mode of delivery
    \item Facility of delivery location, or catchment area of facility
    \item Location of research site
    \item Type of facility (health center/hospital)
    \item Maternal HIV status
    \item Gravidity, parity
    \item Maternal anthropometry (weight, height, BMI, MUAC)
\end{itemize}

\textbf{Associated metadata/documentation}
\begin{itemize}
    \item Study protocol
    \item Codebooks
    \item Do files
    \item Documentation on definitions, components and processing of the data
\end{itemize}

\textbf{Purpose of Data Transfer:} The data will be used to quantify the current and future impacts of heat exposure on maternal and child health in sub-Saharan Africa.

\textbf{Data Source 2:}

[Repeat as above for each data set to be shared]

\newpage

\section*{ANNEXURE B:}

\textbf{DESCRIPTION OF STUDY}

\textbf{Study title:} Individual Participant Data meta-analysis to quantify the impact of high ambient temperatures on maternal and child health in Africa

\textbf{Study rationale:} Global temperatures have already increased by 1.1°C since the industrial revolution and are projected to rise by a further 1-2 degrees over the coming decades. Africa is the continent hardest hit by climate change and temperatures are rising at twice the global rate in many parts of the continent.

The harmful impacts of extreme heat on health are well recognised, affecting a range of population groups, including pregnant women and children. There remain, however, major gaps in evidence on the size of temperature impacts, and which outcomes are most affected. Gaps in evidence are especially large in Africa. A study drawing together the rich data collected in trials and cohorts across the continent could provide the information needed to develop solutions to this rapidly escalating public health problem.

An Individual Participant Data (IPD) meta-analysis entails systematically locating, appraising, transforming, and analysing participant-level data from multiple studies which have a common outcome of interest. Unlike classic systematic reviews which use aggregated study-level data extracted from a publication, an IPD involves analyses of raw participant-level data from multiple studies. This approach can overcome many of the biases of classic systematic reviews, and the challenges in understanding heterogeneity and methodological diversity across published studies.

Analysing pooled participant-level data from multiple settings and time periods also holds several notable advantages over analyses of individual databases from a single location and time, most especially through increasing statistical power and generalisability.

The IPD forms parts of the HE\textsuperscript{2}AT Center (HEat and HEalth African Transdisciplinary Center) which consists of partners from South Africa (University of Cape Town and Wits Health Consortium, and IBM-Research Africa), Côte d'Ivoire (University of Peleforo Gon Coulibaly), Zimbabwe (CeSHHAR), and the United States (Universities of Michigan and Washington). The Center is funded through the United States NIH Harnessing Data Science for Health Discovery and Innovation in Africa (DS-I Africa) program. DS-I Africa aims to make optimum use of existing data resources across Africa to address the most pressing health concerns on the continent.

\textbf{Study objectives}: The overall objective of the study is to use innovative data science approaches to quantify the current and future impacts of heat exposure on maternal and child health in sub-Saharan Africa.

The specific objectives are:
\begin{enumerate}
    \item To locate, acquire, collate and transform prospectively collected data from cohort studies and randomized trials on maternal and child health in sub-Saharan Africa.
    \item To develop a collaboration between the HE\textsuperscript{2}AT Center and investigators of each of the studies who contribute participant-level data.
    \item To link health outcome data spatially and temporally with weather and other environmental data, as well as with socio-economic and related factors.
    \item To utilize classic statistical methods and novel machine learning approaches to understand and quantify the impact of heat exposure on maternal and child health.
    \item To document variations in the relationship between heat exposure, and other environmental data, and maternal and child health outcomes across different settings, climate zones and population groups in sub-Saharan Africa.
\end{enumerate}

\textbf{Methods:} Full details of the study have been published in the BMJ Open journal\footnote{Chersich MF, Jack C, Chibwesa C, Ebi KL, Makin JD, Mageto F, Cresswell JA, Luchters S, Wanjiru EG, Asiki G, Alam N, Ngwenya B, Temmerman M, Patz JA, Wafula SW, Vounatsou P, Giorgi E, Walker DM, Waljee AK, Fong O, Koné B, Cissé G; Global Research Alliance for Sustainable Finance and Investment (GRASFI). Measuring the effects of temperature on maternal, newborn and child health in Africa: a protocol for a multi-site time-series study. BMC Pregnancy Childbirth. 2023 Jun 3;23(1):389. doi: 10.1186/s12884-023-05708-0. PMID: 37271483; PMCID: PMC10240989.} and are summed here. We will systematically locate eligible studies through a mapping review of publication databases, the searching of data repositories, and through communicating with experts in the field. Eligibility is based on study- and individual-level criteria. To be eligible, the study needs to include longitudinal data, have enrolled or plan to enrol at least 1000 pregnant women in sub-Saharan Africa, have collected data on key maternal and child outcomes and be identified in publications between January 2012 and June 2022 or through other means such as trial registries or suggestions from other researchers. At an individual level, participants need to have been recruited during pregnancy or intrapartum and have data available on date and location of childbirth. For studies with no date of childbirth, data should be available on date and location of diagnosis of an adverse maternal health outcome, or end of pregnancy in cases of maternal deaths or abortion. Location information may include facility of birth, or city of the study, for example. The datasets from individual studies will be harmonised through the recoding of raw individual participant data into a common set of variables. Various traditional statistical models such as time-series analysis, time-to event analysis and generalised additive models, as well as novel machine learning approaches will be used to quantify associations between high ambient temperatures, and adverse maternal and child outcomes. Data analysis occurs in several stages. Firstly, each study will be analysed individually. Then, data from the individual studies are aggregated to provide a pooled estimate of effect. If heterogeneity between studies is high, then aggregation across studies may not be done, or may only be done in particular groups of studies that share common characteristics.

\textbf{Ethical and legal considerations}: The study has received ethics approval from the Human Research Ethics Committee of the University of the Witwatersrand, South Africa (Ref. No. 220605). There is minimal risk to individual study participants. Participant privacy will be protected as far as possible through the removal of participant identifiers before data transfer, data encryption, and security measures such as limiting the personnel who have access to data, and data storage in secure, password-protected servers. Data sharing across countries can involve legal considerations depending on legislation in particular countries.

\textbf{PROSPERO registration}: PROSPERO 2022 CRD42022346068

\textbf{Funding acknowledgement}: The study is funded by the Fogarty International Center and National Institute of Environmental Health Sciences (NIEHS) and OD/Office of Strategic Coordination (OSC) of the National Institutes of Health under Award Number U54 TW 012083.

\newpage

\section*{ANNEXURE C:}

\textbf{HEAT CENTER CONSORTIUM MEMBERS AS AT DATE OF SIGNING THIS DATA TRANSFER AGREEMENT:}

\begin{itemize}
    \item Wits Health Consortium (Pty) Ltd, South Africa*
    \item University of Cape Town, South Africa*
    \item International Business Machines (IBM) Corporation through its Thomas J. Watson Research Center, USA*
    \item University of Peleforo Gon Coulibaly, Côte d'Ivoire*
    \item Centre for Sexual Health and HIV AIDS Research (CeSHHAR), Zimbabwe*
    \item University of Michigan, United States
    \item University of Washington, United States
\end{itemize}

*Only these HE\textsuperscript{2}AT Center Consortium Members shall have access to the Consortium-Shared Data for purposes of the RP1 Study analysis

\newpage

\section*{ANNEXURE D:}

\textbf{AUTHORSHIP GUIDELINES FOR STUDIES WHO CONTRIBUTE DATA}

Study Principal Investigators, Site Principal Investigators, and additional contributing study members will be invited to be part of the authorship group for any publications that include use of the data from their study.

The authorship guidelines adhere to the ICMJE criteria for authorship, which include:
\begin{enumerate}
    \item Substantial contributions to the conception or design of the work; or the acquisition, analysis, or interpretation of data for the work; AND
    \item Drafting the work or revising it critically for important intellectual content; AND
    \item Final approval of the version to be published; AND
    \item Agreement to be accountable for all aspects of the work in ensuring that questions related to the accuracy or integrity of any part of the work are appropriately investigated and resolved.
\end{enumerate}

The authorship guidelines and study acknowledgements are based on an appreciation of the substantial contribution made by Principal Investigators in providing data from their study, and in recognition of the work involved in conducting the study.

We will include one author per included study (usually study PI), but additional country-PI will be included for multi-country studies. The listed authors of the studies which are contributing data will be named in alphabetical order by surname, from positions 4\textsuperscript{th} author to second-last author. As such, authorships 1-3 and last authorship will be reserved for those who contributed most to the work, and as per ICMJE.

Some journals may place a restriction on the number of authors that may be listed and require that additional authors beyond that number should be included as part of the '*HE\textsuperscript{2}AT Center Study Group*'. In this situation, the HE\textsuperscript{2}AT Center Steering Committee will have the right to make a decision on final authorship, taking into consideration the studies which contributed most participants to the IPD.

The study group will be published in an Appendix where journals will allow this, or otherwise be listed in the acknowledgement section. Here, listing will be done by role in the study and/or by Study/site. Any additional contributors from a study, who adhere to ICMJE criteria will be listed as part of the '*HE\textsuperscript{2}AT Center Study Group*' in an Appendix where journals will allow this, or otherwise be listed in the acknowledgement section.

The name of the funder of the contributing study and of other Principal Investigators will be included in the acknowledgements, as relevant.

Study Principal Investigators may be granted access to the RP1 De-Identified Data for secondary analyses, provided they complete the Data Request Forms, which will then be reviewed by the Data Access Committee (DAC). Decisions around data access are governed by the HE²AT Center's Data Management Plan and the Publication Policy Standard Operating Procedures.

\end{document}