\documentclass[12pt,letterpaper]{article}
\usepackage[margin=1in]{geometry}
\usepackage{xcolor}
\usepackage{ulem}
\usepackage{titlesec}
\usepackage{enumitem}
\usepackage{fancyhdr}
\usepackage{lastpage}
\usepackage{setspace}
\definecolor{deletecolor}{RGB}{255,0,0}
\definecolor{addcolor}{RGB}{0,0,255}
\newcommand{\deleted}[1]{\textcolor{deletecolor}{\sout{#1}}}
\newcommand{\added}[1]{\textcolor{addcolor}{#1}}
\pagestyle{fancy}
\fancyhf{}
\renewcommand{\headrulewidth}{0pt}
\fancyfoot[C]{Page \thepage\ of \pageref{LastPage}}
\titleformat{\section}{\normalfont\Large\bfseries}{\thesection}{1em}{}
\titleformat{\subsection}{\normalfont\large\bfseries}{\thesubsection}{1em}{}
\begin{document}
\onehalfspacing
\begin{center}
\textbf{\Large DATA SHARING AGREEMENT}
\end{center}
\textbf{LEGEND:} \deleted{Red strikethrough text} = deleted text; \added{Blue text} = added text
\vspace{0.5cm}
\hrule
\vspace{0.5cm}
WITS HEALTH CONSORTIUM (PTY) LTD Registration Number: 1997/15443/07 31 Princess of Wales Terrace, Parktown, Johannesburg, 2193, South Africa

(hereinafter “the Data Recipient”)

WHEREAS:

1.	The Data Provider collected certain Data (as defined below) under the following projects: 1.1	[Description of project 1] 1.2	[Description of project 2]

2.	The Data Recipient is carrying out a project titled “Developing Data Science Solutions to Mitigate the Health Impacts of Climate Change in Africa: the HE2AT Center” (“HE2AT Project”) which is funded by the National Institutes of Health (NIH).  

3.	The Data Recipient has requested the Data Provider to transfer certain Data that was collected under the Project/s listed in Clause 1 for purposes of the “Individual Participant Data meta-analysis to quantify the impact of high ambient temperatures on maternal and child health in Africa” (“the Study”) within the HE2AT Project, the details of which are set out under Annexure “B” attached hereto.  

4.	The Data Provider will transfer a Limited Data Set to the Data Recipient. A “Limited Data Set” consists of health information that has had all direct identifiers concerning the subject of the record (and his or her employer, family, and household members) deleted; that is, the information excludes all of the following: names; street addresses (excluding suburb, small area or town); telephone numbers; fax numbers; electronic mail addresses; government insurance numbers; medical record numbers; health plan beneficiary numbers; account numbers; certificate/license numbers; vehicle identifiers and serial numbers, including license plate numbers; device identifiers and serial numbers; web universal resource locators (URLs); internet protocol (IP) address numbers; biometric identifiers, including finger and voice prints; and full-face photographic images and any comparable images.

5.	The transfer of the Data will be done in accordance with the terms and conditions of this Agreement. 

THEREFORE, THE PARTIES AGREE AS FOLLOWS:

1.	DEFINITIONS

    In this Agreement, unless the context otherwise indicates, the following words will have the following meanings:

1.1	"the/this Agreement" shall mean this Agreement together with any Annexures hereto;

1.2	"Commencement Date" shall mean the date on which this Agreement commenced, namely [commencement date];  

\added\{"Amendment Effective Date" shall mean the date on which this Amendment becomes effective and binding upon the Parties and shall be the date of signature of the last Party to sign this Amendment. For the avoidance of doubt, the Amendment Effective Date shall be the same as the effective date of this Agreement unless otherwise specified.\}

\added\{"Azure Cloud Platform" means the Microsoft Azure cloud computing service that will serve as the HE²AT Center Primary Repository.\} \added\{"Cloud Migration" means the process of transferring the Data and any derived data sets from the existing on-premises infrastructure to the Azure Cloud Platform.\} \added\{"Data Access Committee or DAC" means the committee established by the HE²AT Center to review and approve data access requests from external researchers according to established criteria and protocols, which shall continue to function after the conclusion of the HE²AT Center Project.\} \added\{"Data Access Levels" means the tiered access system implemented by the HE²AT Center consisting of: - Level 0: Original Study Data - Raw, unprocessed data with restricted access to Core Data Team only - Level 1: Consortium Shared Data - Processed data shared only among HE²AT Center Consortium partners - Level 2: RP1/RP2 De-identified Data - Retained by HE²AT Center for approved external researcher access - Level 3: Inferential Data - Aggregated and anonymized data available for open access\} \added\{"Extended Research Use" means the use of data derived from the Data by External Researchers for scientific research purposes that may extend beyond but remain consistent with the original aims of the HE²AT Center Project.\} \added\{"External Researcher" means any qualified researcher who is not a member of the HE²AT Center Consortium but who has been approved by the Data Access Committee to access Level 2 data for specific research purposes.\} \added\{"Geo-Distributed Storage Architecture" means the cloud architecture that enables data storage in specific geographic regions in compliance with applicable data protection laws.\} \added\{"Post-Project Data Repository" means the secure data repository that will maintain and govern access to the data after the conclusion of the HE²AT Center Project.\} \added\{"Post-Project Data Use" means the continued storage, access, and use of the data after the conclusion of the HE²AT Center Project in accordance with this Amendment.\} \added\{"Successor Governance Entity" means any entity or institution that assumes responsibility for the governance, maintenance, and oversight of the Post-Project Data Repository after the conclusion of the HE²AT Center Project.\}

1.3	“Responsible Party” means a public or private body or any other person which, alone or in conjunction with others, determines the purpose of and means for Processing Personal Data;

1.4	 “Data” shall mean the Data to be transferred from the Data Provider to the Data Recipient as described and detailed in Annexure A;

1.5	“Data Protection Legislation” shall mean any data protection or data privacy laws applicable, including but not limited to POPIA, the Electronic Communications and Transactions Act 26 of 2005, the Consumer Protection Act 68 of 2008, and the General Data Protection Regulation (GDPR).

1.6	“Data Subject” means the person to whom Personal Data relates;

1.7	“Parties" shall mean the parties to this Agreement, namely Wits Health Consortium (Pty) Ltd and [provider institution]; and the term “Party” shall refer to either of them;

1.8	“person” means a natural or juristic person;

1.9	“Personal Data” means information relating to an identifiable, living, natural person, and where it is applicable, an identifiable, existing juristic person. Key coded data are considered Personal Data even if the holder of that data does not have access to the key that links the data to the identity of an individual;	

1.10	“Processing” (or its conjugates) shall mean any operation or set of operations, which is performed upon Personal Data, whether or not by automatic means, such as collection, recording, organization, storage, adaptation or alteration, retrieval, consultation, use, disclosure by transmission, dissemination or otherwise making available, alignment or combination, blocking, erasure or destruction.

1.11	“Operator” means a person who processes Personal Data for a Responsible Party in terms of a contract or mandate, without coming under the direct authority of that party;

1.12	"the Project / HE2AT Project" shall mean the project entitled “Developing Data Science Solutions to Mitigate the Health Impacts of Climate Change in Africa: the HE2AT Center” funded by the National Institutes of Health;

1.13	“Study” shall mean the specific study within the Project as more fully described in Annexure “B” attached hereto;

1.14	“Study Data” shall mean data and results produced in the execution of the Study; 

1.15	“POPIA” shall mean the South African Protection of Personal Information Act 4 of 2013 and regulations as amended from time to time;

1.16	Words importing the singular shall include the plural and vice versa, and words importing the masculine gender shall include females. The head notes to the clauses to this Agreement are inserted for reference purposes only and shall not affect the interpretation of any of the provisions to which they relate.

2.	TRANSFER AND USE OF DATA

2.1	This Agreement shall commence on the Commencement Date and will terminate on                      30 June 2026 or upon completion of the Project whichever event occurs first.

2.2	Notwithstanding the abovementioned, either Party may cancel this Agreement with 30 (thirty) days’ prior written notice.  On termination of this Agreement, the Data Recipient will immediately discontinue use of the Data and will return all copies of same to the Data Provider or alternatively, and on the Data Provider’s written instruction, destroy all copies of the Data. The Data Provider however acknowledges that in order to maintain the integrity of results from the Project, the ability to amend, restrict, or delete Data disclosed to Data Recipient may be limited, in accordance with applicable regulations.

2.3	Subject to the terms and conditions of this Agreement, Data Provider grants the Data Recipient the non-exclusive right to use the Data solely for purposes of the Study and/or HE2AT Project, for the duration of this Agreement. 

2.4	Each Party shall pay its own costs incurred in the performance of this Agreement. Any given expense or cost can only be committed in writing by the Party responsible for the cost in question. In no case can one Party commit an expense on behalf of another Party, without prior written consent.

2.5	\deleted\{\deleted\{Data Provider retains ownership of the Data and retains all rights to distribute the Data to other third parties. Data Provider warrants its authority and that it has obtained the necessary consent required to provide the Data to the Data Recipient.\}\} \added\{Data Provider retains ownership of the Data and retains all rights to distribute the Data to other third parties. The Data Provider hereby grants the Data Recipient a perpetual, irrevocable, worldwide, non-exclusive license to: (a) store the Data in the Azure Cloud Platform as part of the HE²AT Center's Cloud Migration; (b) process and transform the Data to create derived data sets at Levels 1, 2, and 3; (c) retain the derived data sets for Post-Project Data Use as specifically authorized in this Amendment; and (d) grant access to Level 2 data to External Researchers in accordance with the procedures set forth in this Amendment. This expanded license does not transfer ownership of the Data, which remains with the Data Provider.\}

\added\{The Data Provider acknowledges and agrees that initially the Data shall be accessible only to the Core HE²AT Center Data Management Team for purposes of pre-processing, harmonisation and integration to produce Consortium Shared Data as set out in the HE²AT Center Data Management Plan. Following the Cloud Migration, the Data will be classified as Level 0 data in the Azure Cloud Platform's tiered data access system and will remain accessible only to the Core HE²AT Center Data Management Team.\}

<span style='color: blue'>2.X Cloud Storage Infrastructure and Data Migration Authorization

2.X.1 The Data Provider hereby irrevocably authorizes the Data Recipient to: (a) migrate the Data from on-premises infrastructure to the Azure Cloud Platform; (b) store and process the Data and all derived data sets in the Azure Cloud Platform; and (c) implement the tiered Data Access Levels system described in this Amendment.

2.X.2 The Data Recipient shall implement and maintain the following security and compliance measures in the Azure Cloud Platform: (a) AES-256 encryption for data at rest; (b) TLS encryption for data in transit; (c) Azure Active Directory Role-Based Access Control; (d) Conditional Access Controls; (e) Azure Key Vault for key management; (f) Continuous Monitoring (Azure Monitor); (g) Audit Logging (Log Analytics); (h) Threat Detection (Azure Sentinel).

2.X.3 The Data Recipient shall implement and maintain appropriate data security methods according to the Data Access Level, including but not limited to: (a) Level 0: Safe Harbor approach, Expert determination (b) Level 1: POPIA/GDPR compliance, Role-Based Access Control (c) Level 2: Location jittering, Population density-aware spatial k-anonymity (d) Level 3: Geographic aggregation to census areas

2.X.4 The Data Recipient shall maintain audit trails for all data access events regardless of access level.

2.X.5 The Data Recipient shall store data in compliant regions to prevent unlawful data transfer and shall implement appropriate geographic boundaries for data residency.</span>

2.6	The Data Provider will transfer the Data as is without any warranties, express or implied, including without limitation, any warranty of fitness for a particular purpose. This Agreement does not grant any rights, license or other proprietary interest to the Data Recipient in the Data save as provided for in this Agreement. 

2.7	Data Recipient will use the Data only for purposes of the Project.  If the Data Recipient seeks to use Data for other purposes, the Data Recipient will obtain written consent from Data Provider, either by an amendment to this Agreement or a new agreement, before such use. The Data Recipient will report to the Data Provider on the results of the Project or Study stemming from the use of the Data. 

2.8	The Data Recipient is hereby authorised to transfer the Data to the following third parties listed below (“Collaborators”) for purposes of the Project:     2.8.1	University of Peleforo Gon Coulibaly, Côte d'Ivoire     2.8.2	CeSHHAR, Zimbabwe     2.8.3	IBM Research Africa      2.8.4	University of Cape Town     and subject to the Data Recipient and the relevant Collaborator/s entering into a Data Transfer Agreement on the same terms as provided for herein.  

2.9	The Data Recipient undertakes not to attempt to identify the Data Subject to whom the Data relates.

2.10	The Parties acknowledge their obligation(s) to comply with Data Protection Legislation and that violation of the Data Protection Legislation may subject them to applicable legal penalties.

2.11	If any publications emanate from the use of the Data, the Data Recipient undertakes not to publish the Data in an identifiable form. 

2.12	Under NIH grant funding policy, Study Data resulting from analysis of the Data will, where no personally identifiable data is included, be made openly available through open data access platforms to support further research.

\added\{The Data Provider explicitly acknowledges, agrees, and authorizes that Level 2 (RP1/RP2 De-identified Data) and Level 3 (Inferential Data) will be retained by the HE²AT Center and/or its Successor Governance Entity in perpetuity for potential future projects subject to review and approval by the Data Access Committee and in accordance with all applicable ethical approvals.\}

2.13	\deleted\{\deleted\{Publications emanating from the use of the Data will follow the HE2AT Center authorship policy included in Annexure “C” attached hereto. The HE2AT Center Authorship Policy may be updated from time to time, which updates will be shared between the Parties to this Agreement.\}\} <span style='color: blue'>2.Y External Researcher Access Authorization

2.Y.1 The Data Provider hereby explicitly and irrevocably authorizes and grants permission for appropriately de-identified data derived from the Data (Level 2 Data) to be made available to External Researchers who are not members of the HE²AT Center Consortium, subject to the following conditions: (a) All External Researcher access requests must be reviewed and approved by the Data Access Committee according to established criteria; (b) External Researchers must sign a legally binding Data Use Agreement that restricts their use of the data to the specific approved research purpose and prohibits any attempt to re-identify individuals; (c) External Researchers must commit to appropriate citation of both the HE²AT Center and the original data sources in any publications; (d) External Researchers will only have access to Level 2 data (de-identified) and never to Level 0 or Level 1 data; (e) All External Researcher access will be monitored and logged, with periodic audits conducted to ensure compliance with usage terms; (f) The Data Access Committee shall maintain the right to revoke access for any External Researcher who violates the terms of their Data Use Agreement.

2.Y.2 The authorization for External Researcher access granted in Section 2.Y.1 shall extend beyond the conclusion of the HE²AT Center Project and shall continue in perpetuity as part of the Post-Project Data Use authorized in Section 2.Z.

2.Y.3 The Data Recipient shall provide to the Data Provider, upon request but not more than once annually, a summary report of all External Researcher access that has been granted to data derived from the Data Provider's Data.</span>

2.14	\deleted\{\deleted\{The Data Recipient will retain a copy of the Data for a period of 5 years after the termination of the over-arching NIH grant agreement (current Project End Date 30 June 2026) for the purposes of concluding and correcting any analysis and publications resulting from the Data.  Any retention of Data after this 5 year period will be negotiated with the Data Provider.\}\} \added\{The Data Recipient may retain the Data and derived data in accordance with the HE²AT Center Data Management Plan and the Post-Project Data Use provisions in Section 2.Z of this Amendment. Any retention of Data beyond the period specified in Section 2.Z.1(a) will require further written agreement with the Data Provider. For clarity and the avoidance of doubt, no further authorization or permission shall be required from the Data Provider for the retention and use of Level 1, Level 2, and Level 3 data as specified in Section 2.Z.\}

2.15	By signing this Agreement, the Data Provider confirms that it has the authority to transfer the Data and consent to provide the Data to the Recipient for use for the duration of this Agreement and as provided for in Clause 2.14.

3.	RESPONSIBLE PARTY STATUS	

3.1	For purposes of this Agreement, the Data Recipient is the Responsible Party and the Data Provider is neither the Responsible Party nor an operator. 

3.2	Further, nothing in this Agreement is intended to affect Data Provider’s Processing of Personal Data of Data Subjects unrelated to this Agreement.  Data Provider will not provide any encryption key that could be used to re-identify the patient from any Data provided to Data Recipient.

4.	COMPLIANCE	

Each Party will comply with Data Protection Legislation in relation to the performance of its obligations under this Agreement.

5.	RIGHTS OF DATA SUBJECTS	 

The Parties agree that, as between them, Data Provider is best able to manage requests from Data Subjects for access, amendment, transfer, restriction, or deletion of Personal Data.  In the ordinary course, Data Recipient does not process sufficient information to link Data to an identified individual who makes a request for access, amendment, transfer, or deletion of Personal Data.  In the event that the Data Recipient receives a request from a Data Subject for such access, amendment, transfer, restriction, or deletion, the Data Recipient shall forward the request to Data Provider.  In the event that the Data Provider receives a request from a Data Subject that affects the Data disclosed to the Data Recipient or the Data Recipient’s ability to use or process such Data, Data Provider shall promptly, and no later than five (5) business days notify Data Recipient. Data Provider acknowledges that in order to maintain the integrity of results from the Project, the ability to amend, restrict, or delete Data disclosed to Data Recipient may be limited, in accordance with applicable regulations.

6.	DATA SUBJECT WITHDRAWAL	

Data Recipient acknowledges that Data Subjects may withdraw their informed consent to the Processing of Personal Data at any time.  Data Provider shall promptly notify Data Recipient of any such withdrawal upon which the Data Recipient will immediately discontinue use of the Data Subject’s Personal Data.

7.	CROSS-BORDER DATA TRANSFERS	

7.1	In the event that it is necessary for the Data Recipient to transfer Personal Data across national borders to authorised Collaborator/s or other authorized third parties (as may be agreed between the Parties), the Party providing the Data will ensure the lawful export of the Personal Data and shall enter into a separate agreement governing such transfer on terms no less stringent than the terms set out herein. 

7.2	In the event that the Data is transferred to a jurisdiction where POPIA does not apply, the respective Party transferring the Data undertakes that the Data will only be transferred to a jurisdiction with adequate protection as set out under Section 72 (1) of POPIA. 

8.	SAFEGUARDS	

8.1	Data Recipient will maintain a comprehensive privacy and security program designed to ensure that Personal Data will be used only in accordance with this Agreement or as required by applicable regulations, including the appointment of a Data Protection Officer.  Data Recipient will apply adequate, commercially reasonable technical, physical, and administrative safeguards to protect the Personal Data.  

8.2	Such safeguards shall be appropriate to the nature of the information to prevent any breach of security leading to the accidental or unlawful destruction, loss, alteration, unauthorized disclosure of, or access to Personal Data or any other unauthorized or unlawful use, access, alteration, loss, or disclosure of Personal Data relating to this Agreement (collectively, “Security Breach”).  Data Recipient will also implement appropriate internal policies, procedures, or protocols to minimize the risk of occurrence of a Security Breach.

8.3	Once the Data has been transferred to the Data Recipient, the Data Recipient shall, in line with all applicable legislation and regulations, maintain a comprehensive privacy and security program to ensure the safekeeping and integrity of the Data.

9.	SECURITY BREACH	

9.1	Data Recipient shall notify Data Provider within twenty-four (24) hours of discovery of a potential or actual Security Breach.  In the course of notification, Data Recipient will provide feasible, sufficient information for Data Provider to assess the Security Breach.  Data Provider will determine, in consultation with Data Recipient, if notification to Data Subjects and/or government authorities is required by applicable regulations.  Where Data Provider determines that notification is required by applicable regulations, Data Recipient shall be responsible for all costs and expenses associated with the provision of such notifications.  Data Recipient will also take immediate steps to consult with Data Provider in good faith in the development of remediation efforts to rectify or mitigate the Security Breach.  

9.2	Data Recipient will undertake remediation efforts at its sole expense or will reimburse Data Provider for Data Provider’s reasonable expenses incurred in connection with Data Provider-performed remediation efforts.  In addition to any method of notice described in this Agreement, notice to Data Provider of any Security Breach shall also be reported to \_\_\_\_\_\_\_\_\_\_\_\_\_\_\_\_\_; Telephone:  \_\_\_\_\_\_\_\_\_\_\_\_\_ or Email: \_\_\_\_\_\_\_\_\_\_\_\_\_  

10.	PERSONNEL OBLIGATIONS	

The Parties shall ensure that their respective personnel engaged in the Processing of Personal Data are informed of the confidential nature of the Personal Data, have received appropriate training on their responsibilities, and have executed written confidentiality agreements or are otherwise subject to professional obligations of confidentiality.  The Parties shall ensure that access to Personal Data is limited to those personnel who perform services in accordance with this Agreement.

11.	RECORDS / DATA PROCESSING REGISTER

Data Recipient shall maintain a written record of all Processing activities that are carried out under this Agreement.  Such record shall contain, at a minimum, (i) the name and contact details of any Operators; (ii) the name and contact details of the Operators’ data protection officers; (iii) the categories of Processing that are carried out; (iv) transfers to other countries or international organizations and documentation of the suitable safeguards that are employed; and (v) a general description of the administrative, technical, and physical security measures that have been taken to safeguard the Personal Data.  Data Recipient shall provide Data Provider with a copy of such records upon request.

12.	GOVERNMENT INSPECTIONS

Data Recipient agrees to promptly, and in no case later than five (5) business days, notify Data Provider of any inspection or audit by a government authority concerning compliance with applicable regulations governing the Processing of Personal Data to the extent related to this Agreement.

13.	DATA PROTECTION IMPACT ASSESSMENT		

Data Recipient shall develop and maintain a data protection impact assessment regarding the Processing of Personal Data under this Agreement.  Data Provider shall cooperate with and assist Data Recipient in the development of the data protection impact assessment and/or with prior consultations with government authorities that may be required.

14.	NOTICES

Notices under this Agreement will be given by personal delivery, certified mail, or recognized overnight courier service to the person designated below:

If to Data Recipient Principal Investigator:

Attention: Matthew Francis Chersich (Research Professor)  Climate and Health Directorate, Wits RHI 22 Esselen Street, Hillbrow, Johannesburg 2100  Email: mchersich@wrhi.ac.za 

If to Data Recipient (Legal):

Attention: Alfred Farrell (CEO) Wits Health Consortium (Pty) Ltd, 31 Princess of Wales Terrace, Parktown, Johannesburg, 2193 Email: ceo@witshealth.co.za  

If to Data Provider Investigator:

If to Data Provider (Legal): [Provider legal contact details] Attention: Address: Email:

15.	GENERAL

15.1	In no event shall Data Provider be liable for any use by the Data Recipient of Data or Study Data or for any loss, claim, damage, or liability, of any kind or nature, that may arise from or in connection with this Agreement or Data Recipient’s use, handling, or storage of Data.  

15.2	This Agreement does not constitute, grant nor confer any license under any patents or other proprietary interests of one party to the other, except as explicitly stated in this Agreement.

15.3	This Agreement may be amended by written agreement between the Parties.

15.4	This Agreement may be executed in one or more counterparts, each of which shall be deemed an original, but all of which together shall constitute one and the same instrument. A signed copy of this Agreement delivered by electronic transmission shall be deemed to have the same legal effect as delivery of an original signed copy of this Agreement.

DATA PROVIDER: 		DATA RECIPIENT:

By: 		By: 					     (signature)			(signature) Name: \_\_\_\_\_\_\_\_\_\_\_\_\_\_\_\_\_\_\_\_\_\_\_\_\_\_\_\_\_\_\_\_		Name: \_\_\_\_\_\_\_\_\_\_\_\_\_\_\_\_\_\_\_\_\_\_\_\_\_\_\_\_\_ Title: \_\_\_\_\_\_\_\_\_\_\_\_\_\_\_\_\_\_\_\_\_\_\_\_\_\_\_\_\_\_\_\_			Title: \_\_\_\_\_\_\_\_\_\_\_\_\_\_\_\_\_\_\_\_\_\_\_\_\_\_\_\_\_\_                 Date: 		Date: 				\_\_\_\_\_\_	

ANNEXURE A DESCRIPTION OF DATA

Data Source 1

Project Title: [Full research project title]	

Funder: [Original research funding details].		

Data to be transferred: [Description of data to be transferred].  Individual participant data for a limited set of variables from the original dataset/s relating to maternal and child health, which include:

Essential variables: •	Unique ID (study ID and participant ID) •	Date of delivery of the newborn OR date of follow-up for maternal outcomes •	Location, at a minimum: city of delivery, or city of follow-up (data on location of household, birth facility, or study clinic are preferable)

Maternal outcomes: •	Gestational age at delivery •	Premature rupture of the membranes (PROM) •	Antepartum and postpartum hemorrhage •	Hypertensive disorders in pregnancy •	Anaemia in pregnancy •	Adverse events •	Gestational Diabetes Mellitus (GDM) •	Health facility visits  •	Maternal mental health

Fetal, neonatal and child outcomes •	Prematurity (see also gestational age at delivery) •	Mortality (including cause) •	MTCT (mother-to-child transmission of HIV) •	APGAR score •	Infant growth •	Admission to neonatal intensive care units •	Intrauterine growth restriction

Other variables •	Maternal age •	Date of interviews or examination •	Mode of delivery •	Facility of delivery location, or catchment area of facility •	Location of research site •	Type of facility (health center/hospital) •	Maternal HIV status •	Gravidity, parity •	Maternal weight, height, BMI

Associated metadata/documentation •	codebooks •	do files •	documentation on definitions, components and processing of the data

Purpose of Data Transfer: The data will be used to quantify the current and future impacts of heat exposure on maternal and child health in sub-Saharan Africa.

Data Source 2:

[repeat as above for each data set to be shared]

ANNEXURE B

Study title: Individual Participant Data meta-analysis to quantify the impact of high ambient temperatures on maternal and child health in Africa

Study rationale: Global temperatures have already increased by 1.1°C since the industrial revolution and are projected to rise by a further 1-2 degrees over the coming decades. Africa is the continent hardest hit by climate change and temperatures are rising at twice the global rate in many parts of the continent. 

The harmful impacts of extreme heat on health are well recognised, affecting a range of population groups, including pregnant women and children. There remain, however, major gaps in evidence on the size of temperature impacts, and which outcomes are most affected. Gaps in evidence are especially large in Africa. A study drawing together the rich data collected in trials and cohorts across the continent could provide the information needed to develop solutions to this rapidly escalating public health problem.

An Individual Participant Data (IPD) meta-analysis entails systematically locating, appraising, transforming, and analysing participant-level data from multiple studies which have a common outcome of interest. Unlike classic systematic reviews which use aggregated study-level data extracted from a publication, an IPD involves analyses of raw participant-level data from multiple studies. This approach can overcome many of the biases of classic systematic reviews, and the challenges in understanding heterogeneity and methodological diversity across published studies.

Analysing pooled participant-level data from multiple settings and time periods also holds several notable advantages over analyses of individual databases from a single location and time, most especially through increasing statistical power and generalisability.

The IPD forms parts of the HE2AT Center (HEat and HEalth African Transdisciplinary Center) which consists of partners from South Africa (Universities of Cape Town and Witwatersrand, and IBM-Research Africa), Côte d’Ivoire (University of Peleforo Gon Coulibaly), Zimbabwe (CeSHHAR), and the United States (Universities of Michigan and Washington). The Center is funded through the United States NIH Harnessing Data Science for Health Discovery and Innovation in Africa (DS-I Africa) program1. DS-I Africa aims to make optimum use of existing data resources across Africa to address the most pressing health concerns on the continent.

Study objectives: The overall objective of the study is to use innovative data science approaches to quantify the current and future impacts of heat exposure on maternal and child health in sub-Saharan Africa. The specific objectives are: 1.	To locate, acquire, collate and transform prospectively collected data from cohort studies and randomized trials on maternal and child health in sub-Saharan Africa.

2.	To develop a collaboration between the HE2AT Center and investigators of each of the studies who contribute participant-level data.

3.	To link health outcome data spatially and temporally with weather and other environmental data, as well as with socio-economic and related factors.

4.	To utilize classic statistical methods and novel machine learning approaches to understand and quantify the impact of heat exposure on maternal and child health.

5.	To document variations in the relationship between heat exposure, and maternal and child health outcomes across different settings, climate zones and population groups in sub-Saharan Africa.

Methods: We will systematically locate eligible studies through a mapping review of publication databases, the searching of data repositories, and through communicating with experts in the field. Eligibility is based on study- and individual-level criteria. To be eligible, the study needs to include longitudinal data, have enrolled or plan to enrol at least 1000 pregnant women in sub-Saharan Africa, have collected data on key maternal and child outcomes and be identified in publications between January 2012 and June 2022 or through other means such as trial registries or suggestions from other researchers. At an individual level, participants need to have been recruited during pregnancy or intrapartum, and have data available on date and location of childbirth. For studies with no date of childbirth, data should be available on date and location of diagnosis of an adverse maternal health outcome, or end of pregnancy in cases of maternal deaths or abortion. Location information may include facility of birth, or city of the study, for example. The datasets from individual studies will be harmonised through the recoding of raw individual participant data into a common set of variables. Various traditional statistical models such as time-series analysis, time-to event analysis and generalised additive models, as well as novel machine learning approaches will be used to quantify associations between high ambient temperatures, and adverse maternal and child outcomes. Data analysis occurs in several stages. Firstly, each study will be analysed individually. Then, data from the individual studies are aggregated to provide a pooled estimate of effect. If heterogeneity between studies is high, then aggregation across studies may not be done, or may only be done in particular groups of studies that share common characteristics.

Ethical and legal considerations: The study has received ethics approval from the Human Research Ethics Committee of the University of the Witwatersrand, South Africa (Ref. No. 220605). There is minimal risk to individual study participants. Participant privacy will be protected as far as possible through the removal of participant identifiers before data transfer, data encryption, and security measures such as limiting the personal who have access to data, and data storage in secure, password-protected servers. Data sharing across countries can involve legal considerations depending on legislation in particular countries.

PROSPERO registration: PROSPERO 2022 CRD42022346068 Available from: https://www.crd.york.ac.uk/prospero/display\_record.php?ID=CRD42022346068

Funding acknowledgement: The study is funded by the Fogarty International Center and National Institute of Environmental Health Sciences (NIEHS) and OD/Office of Strategic Coordination (OSC) of the National Institutes of Health under Award Number U54 TW 012083.  ANNEXURE C

Authorship guidelines for studies who contribute data  Study Principal Investigators, Site Principal Investigators, and additional contributing study members will be invited to be part of the authorship group for any publications that include use of the data from their study.  The authorship guidelines adhere to the ICMJE criteria for authorship, which include:   1	Substantial contributions to the conception or design of the work; or the acquisition, analysis, or interpretation of data for the work; AND  2	Drafting the work or revising it critically for important intellectual content; AND  3	Final approval of the version to be published; AND  4	Agreement to be accountable for all aspects of the work in ensuring that questions related to the accuracy or integrity of any part of the work are appropriately investigated and resolved.  The authorship guidelines and study acknowledgements are based on an appreciation of the substantial contribution made by Principal Investigators in providing data from their study, and in recognition of the work involved in conducting the study.  We will include one author per included study (usually study PI), but additional country-PI will be included for multi-country studies. The listed authors of the studies which are contributing data will be named in alphabetical order by surname, from positions 4th author to second-last author. As such, authorships 1-3 and last authorship will be reserved for those who contributed most to the work, and as per ICMJE.  Some journals may place a restriction on the number of authors that may be listed and require that additional authors beyond that number should be included as part of the ‘HEAT Center study Group‘. In this situation, the HEAT Center Steering Committee will have the right to make a decision on final authorship, taking into consideration the studies which contributed most participants to the IPD.  The study group will be published in an Appendix where journals will allow this, or otherwise be listed in the acknowledgement section. Here, listing will be done by role in the study and/or by Study/site. Any additional contributors from a study, who adhere to ICMJE criteria will be listed as part of the ‘HEAT Center study Group’ in an Appendix where journals will allow this, or otherwise be listed in the acknowledgement section.  The name of the funder of the contributing study and of other Principal Investigators will be included in the acknowledgements, as relevant.    Study Principal Investigators can be given access to the harmonized database in cases where they intend to conduct a secondary analysis, and are encouraged to submit a concept note of the proposed research question and analysis, should they wish to lead the analysis and/or writing of the paper. All concept notes will be reviewed by the HEAT Center Steering Committee who will make a decision based on the Publication Policy Standard Operating Procedures of the Center. 

\end{document}