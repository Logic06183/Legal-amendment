\documentclass[12pt,letterpaper]{article}
\usepackage[margin=1in]{geometry}
\usepackage{xcolor}
\usepackage{ulem}
\usepackage{titlesec}
\usepackage{enumitem}
\usepackage{fancyhdr}
\usepackage{lastpage}
\usepackage{setspace}
\definecolor{deletecolor}{RGB}{255,0,0}
\definecolor{addcolor}{RGB}{0,0,255}
\newcommand{\deleted}[1]{\textcolor{deletecolor}{\sout{#1}}}
\newcommand{\added}[1]{\textcolor{addcolor}{#1}}
\pagestyle{fancy}
\fancyhf{}
\renewcommand{\headrulewidth}{0pt}
\fancyfoot[C]{Page \thepage\ of \pageref{LastPage}}
\titleformat{\section}{\normalfont\Large\bfseries}{\thesection}{1em}{}
\titleformat{\subsection}{\normalfont\large\bfseries}{\thesubsection}{1em}{}
\begin{document}
\onehalfspacing
\begin{center}
\textbf{\Large DATA TRANSFER AGREEMENT WITH TRACKED CHANGES - RP1 Old}
\end{center}
\textbf{LEGEND:} \deleted{Red strikethrough text} = deleted text; \added{Blue text} = added text
\vspace{0.5cm}
\hrule
\vspace{0.5cm}
\section*{DATA TRANSFER AGREEMENT}

\section*{ENTERED INTO BY AND BETWEEN}

[Provider legal name, description of entity and address details]

(hereinafter “the Data Provider”)

and

Wits Planetary Health Research Division a Division of Wits Health Consortium (Pty) Ltd

Registration Number: 1997/15443/07

31 Princess of Wales Terrace, Parktown, Johannesburg, 2193, South Africa

(hereinafter “the Data Recipient”)

\section*{WHEREAS:}

The Data Provider collected certain Original Study Data (as defined below) under the following studies:

Project 1: [Title of Project to be inserted]

Project 2: [Title of Project to be inserted]

The Data Recipient is a member of the HE2AT Center Consortium carrying out the research project titled “Developing Data Science Solutions to Mitigate the Health Impacts of Climate Change in Africa: the HE2AT Center” (“HE2AT Project”) which is funded by the National Institutes of Health (NIH).

The Data Recipient has requested the Data Provider to transfer the Original Study Data collected by the Data Provider for the study in (i) above for purposes of the Data Recipient using the Original Study Data in the HE2AT Center Research Project 1, titled: “Individual Participant Data meta-analysis to quantify the impact of high ambient temperatures on maternal and child health in Africa” (“RP1 Study”) the details of which are set out in Annexure “B” attached hereto.

The Data Provider has agreed to provide the Original Study Data as set out in Annexure “A” hereto.

The Parties agree that the transfer of the Original Study Data will be done in accordance with the terms and conditions of this Agreement.

\section*{THEREFORE, THE PARTIES AGREE AS FOLLOWS:}

\section*{DEFINITIONS}

<span style='color: blue'> <span style='color: blue'> "Amendment Effective Date" shall mean the date on which this Amendment becomes effective and binding upon the Parties and shall be the date of signature of the last Party to sign this Amendment. For the avoidance of doubt, the Amendment Effective Date shall be the same as the effective date of this Agreement unless otherwise specified.

"Azure Cloud Platform" means the Microsoft Azure cloud computing service that will serve as the HE²AT Center Primary Repository. "Cloud Migration" means the process of transferring the Original Study Data and any derived data sets from the existing on-premises infrastructure to the Azure Cloud Platform. "Data Access Committee or DAC" means the committee established by the HE²AT Center to review and approve data access requests from external researchers according to established criteria and protocols, which shall continue to function after the conclusion of the HE²AT Center Project. "Data Access Levels" means the tiered access system implemented by the HE²AT Center consisting of: - Level 0: Original Study Data - Raw, unprocessed data with restricted access to Core Data Team only - Level 1: Consortium Shared Data - Processed data shared only among HE²AT Center Consortium partners - Level 2: De-identified Data - Retained by HE²AT Center for approved external researcher access - Level 3: Inferential Data - Aggregated and anonymized data available for open access "Extended Research Use" means the use of data derived from the Original Study Data by External Researchers for scientific research purposes that may extend beyond but remain consistent with the original aims of the HE²AT Center Project. "External Researcher" means any qualified researcher who is not a member of the HE²AT Center Consortium but who has been approved by the Data Access Committee to access Level 2 data for specific research purposes. "Geo-Distributed Storage Architecture" means the cloud architecture that enables data storage in specific geographic regions in compliance with applicable data protection laws. "Post-Project Data Repository" means the secure data repository that will maintain and govern access to the data after the conclusion of the HE²AT Center Project. "Post-Project Data Use" means the continued storage, access, and use of the data after the conclusion of the HE²AT Center Project in accordance with this Amendment. "Successor Governance Entity" means any entity or institution that assumes responsibility for the governance, maintenance, and oversight of the Post-Project Data Repository after the conclusion of the HE²AT Center Project. </span> </span>

In this Agreement, unless the context otherwise indicates, the following words will have the following meanings:

1.1	"the/this Agreement" shall mean this Agreement together with any Annexures hereto;

1.2	"Commencement Date" \deleted\{\deleted\{shall mean the date on which this Agreement shall become effective and binding upon the Parties and shall be the date of signature of the last Party to sign this Agreement;\}\} \added\{IN WITNESS WHEREOF, the Parties have executed this Agreement as of the Amendment Effective Date.\}

1.3	“Responsible Party” means a public or private body or any other person which, alone or in conjunction with others, determines the purpose of and means for Processing Personal Data;

“Original Study Data” shall mean the health-related data listed in Annexure “A” hereto and any other data actually transferred by the Data Provider to the Data Recipient under this Agreement;

“Data Protection Legislation” shall mean any data protection or data privacy laws as may be applicable, including but not limited to POPIA, the Electronic Communications and Transactions Act 26 of 2005, the Consumer Protection Act 68 of 2008, and the General Data Protection Regulation (GDPR);

“Data Subject” means the person to whom Personal Data relates;

“RP1 De-identified Data” means data with the following information deleted; (1) information that identifies the Data Subject, (2) information that can be used or manipulated by a reasonably foreseeable method to identify the Data Subject or, (3) information that can be linked by a reasonably foreseeable method to other information that identifies the Data Subject;

“HE2AT Center Data Management Plan” means the data management plan applicable to the RP1 Study as may be amended and updated from time to time by the HE2AT Center Consortium;

“HE2AT Center Consortium” means the consortium members jointly working on the HEAT Center Project, as listed in Annexure “C”, as may be amended from time to time;

“Core HE²AT Center Data Management Team” a group of named personnel within the HE²AT Center Consortium responsible for the initial processing, harmonisation and integration of the Original Study Data;

1.11	“Parties" shall mean the parties to this Agreement, namely the University of Cape Town and [provider institution]; and the term “Party” shall refer to either of them;

1.12	“person” means a natural or juristic person;

1.13	“Personal Data” means any information relating to an identifiable, living, natural person, and where it is applicable, an identifiable, existing juristic person;

1.14	“Processing” (or its conjugates) shall mean any operation or set of operations, which is performed upon Personal Data, whether or not by automatic means, such as collection, recording, organization, storage, adaptation or alteration, retrieval, consultation, use, disclosure by transmission, dissemination or otherwise making available, alignment or combination, blocking, erasure or destruction;

1.15	“Operator” means a person who processes Personal Data for a Responsible Party in terms of a contract or mandate, without coming under the direct authority of that party;

1.16	"HE2AT Project" shall mean the project entitled “Developing Data Science Solutions to Mitigate the Health Impacts of Climate Change in Africa: the HE2AT Center” funded by the National Institutes of Health;

1.17	“RP1 Study” shall mean the specific study under the HE2AT Project titled: “Individual Participant Data meta-analysis to quantify the impact of high ambient temperatures on maternal and child health in Africa” as more fully described in Annexure “B” attached hereto;

1.18	“RP1 Study Data” shall mean all data resulting from processing of the Original Study Data during the RP1 Study, which includes but is not limited to, RP1 De-identified Data and Consortium Shared Data;

1.19	“POPIA” shall mean the South African Protection of Personal Information Act 4 of 2013 and regulations as amended from time to time;

1.20	“Consortium Shared Data” means data that has undergone, initial processing, harmonisation and integration and includes, amongst other variables, a limited set of indirect identifiers that are required for the purposes of conducting the RP1 Study analysis as described in Annexure “B”.

1.21	"Amendment Effective Date" \deleted\{\deleted\{shall mean the date on which this Amendment becomes effective and binding upon the Parties and shall be the date of signature of the last Party to sign this Amendment.\}\} \added\{shall mean the date on which this Agreement shall become effective and binding upon the Parties and shall be the date of signature of the last Party to sign this Agreement;\}

1.21	Words importing the singular shall include the plural and vice versa, and words importing the masculine gender shall include females. The head notes to the clauses to this Agreement are inserted for reference purposes only and shall not affect the interpretation of any of the provisions to which they relate.

\section*{2.	TRANSFER AND USE OF DATA}

2.1	This Agreement shall commence on the Commencement Date and \deleted\{\deleted\{shall terminate on completion of the HE2AT Project\}\} \added\{shall remain in effect in perpetuity with respect to the Post-Project Data Use provisions set forth in Section 2.21 of this Amendment, unless terminated earlier in accordance with the provisions of this Agreement. The Data Provider specifically acknowledges and agrees that the Post-Project Data Use provisions shall survive the termination of the HE²AT Center Project.\}

2.2	Either Party may terminate this Agreement prior to the completion of the HE2AT Project by providing 30 (thirty) calendar days’ prior written notice to the other Party. On early termination of this Agreement, the Data Recipient shall, where possible, immediately discontinue use of the Original Study Data and upon the Data Provider’s instructions, either return all copies of same to the Data Provider, destroy all copies of the Original Study Data, or deal with the Original Study Data in any other manner requested by the Data Provider. Data Provider acknowledges that the ability to retrieve or delete Original Study Data already incorporated into the RP1 Study Data may be limited due to (1) this being impractical or impossible, (2) the need to maintain the integrity of the RP1 Study Data or (3) legal, operational, or regulatory requirements in accordance with applicable law. Where deletion is not possible, the Data Recipient shall, where practicable, anonymize or pseudonymize the Data to minimize any potential risks associated with its retention. The Data Recipient shall inform the Data Provider of any such measures taken.

2.3	Each Party shall pay its own costs incurred in the performance of this Agreement. Any given expense or cost can only be committed in writing by the Party responsible for the cost in question. In no case can one Party commit an expense on behalf of another Party, without prior written consent.

\{\{ ... \}\}

12.5 	The provisions of this Agreement that by their nature are intended to survive termination or expiration of the Agreement shall survive such termination or expiration and shall remain in full force and effect

12.6 The Parties expressly acknowledge and agree that the provisions of Sections 2.20 and 2.21 of this Amendment regarding External Researcher Access and Post-Project Data Use shall survive the termination of the Agreement and the conclusion of the HE²AT Center Project.

\deleted\{\deleted\{12.6\}\} \added\{12.7\}	The Data Recipient hereby acknowledges and accepts that the signatory of this Agreement on behalf of the Data Provider represents and warrants that they have the full authority and capacity to enter into and bind the Data Provider to the terms of this Agreement. The Data Recipient shall not be held liable or responsible in any manner for any breach of authority or lack thereof by the signatory. Should it be determined that the signatory did not possess the requisite authority to execute this Agreement, the Data Recipient shall remain indemnified and free from any and all claims, liabilities, losses, damages, or expenses arising therefrom.

\section*{ANNEXURE A:}

\section*{DESCRIPTION OF DATA}

\{\{ ... \}\} The below list of variables is indicative, and the final variable list shall be finalised and recorded between Data Provider and Data Recipient based on data availability and relevance.

Data Source 1

Project Title: [Full research project title]

Funder: [Original research funding details].

Data to be transferred: Individual participant data for a limited set of variables from the original dataset/s relating to Maternal outcomes and/or fetal, neonatal and child outcomes.

Dataset includes these important variables:

Essential variables:

Unique ID (study ID and participant ID)

Date of delivery of the newborn OR date of maternal outcomes

Location, at a minimum: city of delivery, or city of follow-up (data on location of household, birth facility, or study clinic are preferable)

Maternal outcomes (indicative list):

Gestational age at delivery

Preterm premature rupture of the membranes

Prolonged rupture of membranes

Antepartum and postpartum haemorrhage

Hypertensive disorders in pregnancy

Anaemia in pregnancy

Adverse events

Gestational Diabetes Mellitus

Health facility visits

Maternal mental health

Fetal, neonatal and child outcomes (indicative list)

Prematurity (see also gestational age at delivery)

Mortality (including cause)

Mother-to-child transmission of HIV (MTCT)

APGAR score

Infant growth

Admission to neonatal intensive care units or paediatric ward

Intrauterine growth restriction

Other variables

Maternal age

Date of interviews or examination

Mode of delivery

Facility of delivery location, or catchment area of facility

Location of research site

Type of facility (health center/hospital)

Maternal HIV status

Gravidity, parity

Maternal anthropometry (weight, height, BMI, MUAC)

Associated metadata/documentation

Study protocol

Codebooks

Do files

Documentation on definitions, components and processing of the data

Purpose of Data Transfer: The data will be used to quantify the current and future impacts of heat exposure on maternal and child health in sub-Saharan Africa.

Data Source 2:

[repeat as above for each data set to be shared]

\section*{ANNEXURE B:}

\section*{DESCRIPTION OF STUDY}

Study title: Individual Participant Data meta-analysis to quantify the impact of high ambient temperatures on maternal and child health in Africa

Study rationale: Global temperatures have already increased by 1.1°C since the industrial revolution and are projected to rise by a further 1-2 degrees over the coming decades. Africa is the continent hardest hit by climate change and temperatures are rising at twice the global rate in many parts of the continent.

The harmful impacts of extreme heat on health are well recognised, affecting a range of population groups, including pregnant women and children. There remain, however, major gaps in evidence on the size of temperature impacts, and which outcomes are most affected. Gaps in evidence are especially large in Africa. A study drawing together the rich data collected in trials and cohorts across the continent could provide the information needed to develop solutions to this rapidly escalating public health problem.

An Individual Participant Data (IPD) meta-analysis entails systematically locating, appraising, transforming, and analysing participant-level data from multiple studies which have a common outcome of interest. Unlike classic systematic reviews which use aggregated study-level data extracted from a publication, an IPD involves analyses of raw participant-level data from multiple studies. This approach can overcome many of the biases of classic systematic reviews, and the challenges in understanding heterogeneity and methodological diversity across published studies.

Analysing pooled participant-level data from multiple settings and time periods also holds several notable advantages over analyses of individual databases from a single location and time, most especially through increasing statistical power and generalisability.

<span style='color: blue'>

\subsection*{2.19 Cloud Storage Infrastructure and Data Migration Authorization}

2.19.1 The Data Provider hereby irrevocably authorizes the Data Recipient to: (a) migrate the Original Study Data from on-premises infrastructure to the Azure Cloud Platform; (b) store and process the Original Study Data and all derived data sets in the Azure Cloud Platform; and (c) implement the tiered Data Access Levels system described in this Amendment.

\subsection*{2.20 External Researcher Access Authorization}

2.20.1 The Data Provider hereby explicitly and irrevocably authorizes and grants permission for appropriately de-identified data derived from the Original Study Data (Level 2 Data) to be made available to External Researchers who are not members of the HE²AT Center Consortium, subject to the following conditions: (a) All External Researcher access requests must be reviewed and approved by the Data Access Committee. (b) External Researchers must sign a legally binding Data Use Agreement. (c) External Researchers must commit to appropriate citation of both the HE²AT Center and the original data sources. (d) External Researchers will only have access to Level 2 data (de-identified). (e) All External Researcher access will be monitored and logged. (f) The Data Access Committee shall maintain the right to revoke access for any External Researcher.

\subsection*{2.21 Post-Project Data Use and Long-Term Data Retention Authorization}

2.21.1 The Data Provider hereby explicitly and irrevocably authorizes and grants permission that the data derived from the Original Study Data shall be retained and may continue to be used beyond the conclusion of the HE²AT Center Project as follows: (a) Level 0 Data (Original Study Data): Shall be retained for a period of 5 (five) years. (b) Level 1 Data (Consortium Shared Data): Shall be retained for a period of 10 (ten) years. (c) Level 2 Data (De-identified Data): Shall be retained indefinitely as a scientific resource. (d) Level 3 Data (Inferential Data): Shall be retained indefinitely as an open scientific resource. </span>

The IPD forms parts of the HE2AT Center (HEat and HEalth African Transdisciplinary Center) which consists of partners from South Africa (University of Cape Town and Wits Health Consortium, and IBM-Research Africa), Côte d’Ivoire (University of Peleforo Gon Coulibaly), Zimbabwe (CeSHHAR), and the United States (Universities of Michigan and Washington). The Center is funded through the United States NIH Harnessing Data Science for Health Discovery and Innovation in Africa (DS-I Africa) program. DS-I Africa aims to make optimum use of existing data resources across Africa to address the most pressing health concerns on the continent.

Study objectives: The overall objective of the study is to use innovative data science approaches to quantify the current and future impacts of heat exposure on maternal and child health in sub-Saharan Africa.

The specific objectives are:

To locate, acquire, collate and transform prospectively collected data from cohort studies and randomized trials on maternal and child health in sub-Saharan Africa.

To develop a collaboration between the HE2AT Center and investigators of each of the studies who contribute participant-level data.

To link health outcome data spatially and temporally with weather and other environmental data, as well as with socio-economic and related factors.

To utilize classic statistical methods and novel machine learning approaches to understand and quantify the impact of heat exposure on maternal and child health.

To document variations in the relationship between heat exposure, and other environmental data, and maternal and child health outcomes across different settings, climate zones and population groups in sub-Saharan Africa.

Methods: Full details of the study have been published in the BMJ Open journal1 and are summed here. We will systematically locate eligible studies through a mapping review of publication databases, the searching of data repositories, and through communicating with experts in the field. Eligibility is based on study- and individual-level criteria. To be eligible, the study needs to include longitudinal data, have enrolled or plan to enrol at least 1000 pregnant women in sub-Saharan Africa, have collected data on key maternal and child outcomes and be identified in publications between January 2012 and June 2022 or through other means such as trial registries or suggestions from other researchers. At an individual level, participants need to have been recruited during pregnancy or intrapartum and have data available on date and location of childbirth. For studies with no date of childbirth, data should be available on date and location of diagnosis of an adverse maternal health outcome, or end of pregnancy in cases of maternal deaths or abortion. Location information may include facility of birth, or city of the study, for example. The datasets from individual studies will be harmonised through the recoding of raw individual participant data into a common set of variables. Various traditional statistical models such as time-series analysis, time-to event analysis and generalised additive models, as well as novel machine learning approaches will be used to quantify associations between high ambient temperatures, and adverse maternal and child outcomes. Data analysis occurs in several stages. Firstly, each study will be analysed individually. Then, data from the individual studies are aggregated to provide a pooled estimate of effect. If heterogeneity between studies is high, then aggregation across studies may not be done, or may only be done in particular groups of studies that share common characteristics.

<span style='color: blue'>

\subsection*{2.19 Cloud Storage Infrastructure and Data Migration Authorization}

2.19.1 The Data Provider hereby irrevocably authorizes the Data Recipient to: (a) migrate the Original Study Data from on-premises infrastructure to the Azure Cloud Platform; (b) store and process the Original Study Data and all derived data sets in the Azure Cloud Platform; and (c) implement the tiered Data Access Levels system described in this Amendment.

\subsection*{2.20 External Researcher Access Authorization}

2.20.1 The Data Provider hereby explicitly and irrevocably authorizes and grants permission for appropriately de-identified data derived from the Original Study Data (Level 2 Data) to be made available to External Researchers who are not members of the HE²AT Center Consortium, subject to the following conditions: (a) All External Researcher access requests must be reviewed and approved by the Data Access Committee. (b) External Researchers must sign a legally binding Data Use Agreement. (c) External Researchers must commit to appropriate citation of both the HE²AT Center and the original data sources. (d) External Researchers will only have access to Level 2 data (de-identified). (e) All External Researcher access will be monitored and logged. (f) The Data Access Committee shall maintain the right to revoke access for any External Researcher.

\subsection*{2.21 Post-Project Data Use and Long-Term Data Retention Authorization}

2.21.1 The Data Provider hereby explicitly and irrevocably authorizes and grants permission that the data derived from the Original Study Data shall be retained and may continue to be used beyond the conclusion of the HE²AT Center Project as follows: (a) Level 0 Data (Original Study Data): Shall be retained for a period of 5 (five) years. (b) Level 1 Data (Consortium Shared Data): Shall be retained for a period of 10 (ten) years. (c) Level 2 Data (De-identified Data): Shall be retained indefinitely as a scientific resource. (d) Level 3 Data (Inferential Data): Shall be retained indefinitely as an open scientific resource. </span>

Ethical and legal considerations: The study has received ethics approval from the Human Research Ethics Committee of the University of the Witwatersrand, South Africa (Ref. No. 220605). There is minimal risk to individual study participants. Participant privacy will be protected as far as possible through the removal of participant identifiers before data transfer, data encryption, and security measures such as limiting the personnel who have access to data, and data storage in secure, password-protected servers. Data sharing across countries can involve legal considerations depending on legislation in particular countries.

PROSPERO registration: PROSPERO 2022 CRD42022346068 Available from:https://www.crd.york.ac.uk/prospero/documents/PROSPERO registration form.pdf https://www.crd.york.ac.uk/prospero/display\_record.php?ID=CRD42022346068

Funding acknowledgement: The study is funded by the Fogarty International Center and National Institute of Environmental Health Sciences (NIEHS) and OD/Office of Strategic Coordination (OSC) of the National Institutes of Health under Award Number U54 TW 012083.

\section*{ANNEXURE C:}

\section*{HEAT CENTER CONSORTIUM MEMBERS AS AT DATE OF SIGNING THIS DATA TRANSFER AGREEMENT:}

Wits Health Consortium (Pty) Ltd, South Africa*

University of Cape Town, South Africa*

International Business Machines (IBM) Corporation through its Thomas J. Watson Research Center, USA*

University of Peleforo Gon Coulibaly, Côte d’Ivoire*

Centre for Sexual Health and HIV AIDS Research (CeSHHAR), Zimbabwe*

University of Michigan, United States

University of Washington, United States

*Only these HE2AT Center Consortium Members shall have access to the Consortium-Shared Data for purposes of the RP1 Study analysis

\section*{ANNEXURE D:}

\section*{AUTHORSHIP GUIDELINES FOR STUDIES WHO CONTRIBUTE DATA}

Study Principal Investigators, Site Principal Investigators, and additional contributing study members will be invited to be part of the authorship group for any publications that include use of the data from their study.

The authorship guidelines adhere to the ICMJE criteria for authorship, which include:

Substantial contributions to the conception or design of the work; or the acquisition, analysis, or interpretation of data for the work; AND

Drafting the work or revising it critically for important intellectual content; AND

Final approval of the version to be published; AND

Agreement to be accountable for all aspects of the work in ensuring that questions related to the accuracy or integrity of any part of the work are appropriately investigated and resolved.

The authorship guidelines and study acknowledgements are based on an appreciation of the substantial contribution made by Principal Investigators in providing data from their study, and in recognition of the work involved in conducting the study.

We will include one author per included study (usually study PI), but additional country-PI will be included for multi-country studies. The listed authors of the studies which are contributing data will be named in alphabetical order by surname, from positions 4th author to second-last author. As such, authorships 1-3 and last authorship will be reserved for those who contributed most to the work, and as per ICMJE.

Some journals may place a restriction on the number of authors that may be listed and require that additional authors beyond that number should be included as part of the ‘HE2AT Center Study Group‘. In this situation, the HE2AT Center Steering Committee will have the right to make a decision on final authorship, taking into consideration the studies which contributed most participants to the IPD.

The study group will be published in an Appendix where journals will allow this, or otherwise be listed in the acknowledgement section. Here, listing will be done by role in the study and/or by Study/site. Any additional contributors from a study, who adhere to ICMJE criteria will be listed as part of the ‘HE2AT Center Study Group’ in an Appendix where journals will allow this, or otherwise be listed in the acknowledgement section.

The name of the funder of the contributing study and of other Principal Investigators will be included in the acknowledgements, as relevant.

Study Principal Investigators may be granted access to the RP-1 De-Identified Data for secondary analyses, provided they complete the Data Request Forms, which will then be reviewed by the Data Access Committee (DAC). Decisions around data access are governed by the HE²AT Center’s Data Management Plan and the Publication Policy Standard Operating Procedures.

\section*{DATA PROVIDER  | DATA RECIPIENT}

(signature)  |   

(signature)

Name:  | Name:

Title:  | Title:

Date:  | Date:

\end{document}