\documentclass[12pt,letterpaper]{article}
\usepackage{geometry}
\geometry{margin=1in}
\usepackage{titlesec}
\usepackage{enumitem}
\usepackage{xcolor}
\usepackage{fancyhdr}
\usepackage{hyperref}
\hypersetup{
    colorlinks=true,
    linkcolor=blue,
    filecolor=magenta,
    urlcolor=blue,
}

\pagestyle{fancy}
\fancyhf{}
\renewcommand{\headrulewidth}{0pt}
\fancyhead[C]{HE\textsuperscript{2}AT CENTER}
\fancyfoot[C]{\thepage}

\titleformat{\section}
  {\normalfont\Large\bfseries}{\thesection}{1em}{}
\titleformat{\subsection}
  {\normalfont\large\bfseries}{\thesubsection}{1em}{}

\setlist{noitemsep}

\title{}
\author{}
\date{}

\begin{document}

\begin{center}
\textbf{\LARGE FREQUENTLY ASKED QUESTIONS}\\[0.3cm]
\textbf{\large HE\textsuperscript{2}AT CENTER CLOUD MIGRATION \& EXTENDED DATA USE}
\end{center}

\subsection*{1. Why is the HE\textsuperscript{2}AT Center requesting this amendment?}

This amendment addresses three critical needs:

\begin{enumerate}
\item \textbf{Cloud Migration}: NIH funding requirements necessitate moving data from on-premises servers to the Azure cloud platform, providing enhanced security, improved access controls, and better disaster recovery capabilities.

\item \textbf{External Researcher Access}: To maximize the scientific value of the data, we need clear authorization to share appropriately de-identified data with qualified researchers beyond the original HE\textsuperscript{2}AT Center Consortium, subject to strict oversight.

\item \textbf{Long-Term Data Preservation}: Research data has enduring value beyond the timeframe of a single project. This amendment establishes a framework for responsible, secure long-term data preservation and use.
\end{enumerate}

\subsection*{2. Will I still own my data after signing this amendment?}

Yes. The amendment explicitly states that data ownership remains with you as the original data provider. The amendment grants a license to use the data in specific ways, but does not transfer ownership. You retain all rights to distribute your original data to other third parties.

\subsection*{3. What will happen to my data after the HE\textsuperscript{2}AT Center Project ends?}

The data will be handled according to a tiered approach:

\begin{itemize}
\item \textbf{Level 0 (Original Study Data)}: Retained for 5 years after project conclusion for verification purposes, then deleted unless you explicitly agree to a longer retention period.

\item \textbf{Level 1 (Consortium Shared Data)}: Retained for 10 years after project conclusion for use by former HE\textsuperscript{2}AT Consortium members.

\item \textbf{Level 2 (De-identified Data)}: Retained indefinitely as a scientific resource, accessible to approved external researchers through a governed process.

\item \textbf{Level 3 (Inferential Data)}: Retained indefinitely as an open scientific resource.
\end{itemize}

\subsection*{4. Who will have access to my data after it's migrated to the cloud?}

Access is strictly controlled according to data levels:

\begin{itemize}
\item \textbf{Level 0 (Original Study Data)}: Only the Core Data Team
\item \textbf{Level 1 (Consortium Shared Data)}: Only authorized HE\textsuperscript{2}AT Consortium partners
\item \textbf{Level 2 (De-identified Data)}: External researchers with DAC approval
\item \textbf{Level 3 (Inferential Data)}: Open access (highly aggregated, non-identifiable)
\end{itemize}

\subsection*{5. What security measures will protect my data in the cloud?}

The Azure cloud platform implements enterprise-grade security including:
\begin{itemize}
\item AES-256 encryption for all data at rest
\item TLS encryption for all data in transit
\item Multi-factor authentication and role-based access control
\item Azure Key Vault for secure key management
\item Continuous monitoring and comprehensive audit logging
\item Advanced threat detection and security analytics
\end{itemize}

\subsection*{6. How will external researchers gain access to data derived from my study?}

External researchers must:
\begin{enumerate}
\item Submit a formal application to the Data Access Committee (DAC)
\item Receive DAC approval based on scientific merit and ethical considerations
\item Sign a legally binding Data Use Agreement with specific restrictions
\item Access only Level 2 de-identified data, never the original study data
\item Have their usage monitored and logged
\end{enumerate}

You will receive annual reports summarizing who has accessed data derived from your study.

\subsection*{7. What safeguards prevent misuse of my data by external researchers?}

Multiple protective layers are in place:
\begin{itemize}
\item Researchers only receive de-identified data (Level 2)
\item Access is governed by a legally binding Data Use Agreement
\item All access is logged and monitored
\item Researchers must cite original data sources in publications
\item The DAC can revoke access for any violations
\item Geographic controls prevent unlawful cross-border data transfers
\end{itemize}

\subsection*{8. Who will govern the data after the HE\textsuperscript{2}AT Center Project ends?}

The amendment provides for a "Successor Governance Entity" that will maintain:
\begin{itemize}
\item The Data Access Committee or equivalent oversight structure
\item All security and privacy protections
\item Regular compliance reviews for evolving laws and standards
\item Sustainable funding for ongoing data management
\end{itemize}

\subsection*{9. Can I revoke access to my data in the future?}

The amendment establishes perpetual authorizations for Levels 1-3 data, as continued access is essential for scientific reproducibility and follow-up research. However:

\begin{itemize}
\item Level 0 (Original Study Data) will be deleted after 5 years unless you explicitly approve longer retention
\item You can contact the Data Access Committee with specific concerns about data use
\item All uses must comply with the terms established in the amendment and original agreement
\end{itemize}

\subsection*{10. Does this amendment change how my data will be used for research?}

The core research purposes remain consistent with the original agreement and the HE\textsuperscript{2}AT Center's mission. The amendment expands:

\begin{enumerate}
\item Where the data is stored (in the Azure cloud)
\item Who may access appropriately de-identified versions (qualified external researchers)
\item How long the de-identified data may be retained (indefinitely for Levels 2-3)
\end{enumerate}

All research must still be consistent with the scientific aims of understanding heat-health impacts in Africa.

\subsection*{11. Will my organization be acknowledged in publications resulting from external researcher access?}

Yes. All external researchers must commit to appropriate citation of both the HE\textsuperscript{2}AT Center and the original data sources in any publications. This requirement is embedded in their Data Use Agreement.

\subsection*{12. What happens if there is a data breach after the data is moved to the cloud?}

In the event of a security breach:
\begin{itemize}
\item You will be notified within 24 hours of discovery
\item You will receive regular updates on the investigation and remediation
\item This notification requirement continues as long as any level of your data is retained
\item The Data Recipient will take immediate steps to address the breach
\end{itemize}

\subsection*{13. What if I have concerns about specific future uses of my data?}

While the amendment authorizes broad scientific use of de-identified data, we value our relationship with all data providers. If you have specific concerns about how your data may be used:

\begin{enumerate}
\item Contact the Data Access Committee with your concerns
\item The DAC can implement specific restrictions for your dataset
\item You will receive reports on who has accessed data derived from your study
\end{enumerate}

\subsection*{14. Why is "perpetual" authorization necessary for Levels 2-3 data?}

Perpetual authorization for de-identified data is essential for:
\begin{itemize}
\item Scientific reproducibility and verification of findings
\item Enabling follow-up research and meta-analyses
\item Maximizing the long-term value of data collected with public funds
\item Allowing emerging analytical techniques to extract new insights
\end{itemize}

Remember that Level 2-3 data is thoroughly de-identified, with no direct identifiers and geographic jittering or aggregation to prevent re-identification.

\subsection*{15. Who can I contact if I have additional questions?}

\textbf{HE\textsuperscript{2}AT Center Data Steward:}\\
[NAME]\\
[EMAIL]\\
[PHONE]

\textbf{Legal Contact:}\\
[NAME]\\
[EMAIL]\\
[PHONE]

We're happy to schedule a call to discuss any aspects of this amendment in detail.

\end{document}